\chapter{مروری بر کارهای گذشته، روش‌ها و ابزارهای موجود}
\section{مدل‌های کیفیتی}
با یک نگاه اجمالی بر منابعی همچون
\cite{wagner_software_2013}،
\cite{seffah_usability_2006}، 
\cite{pressman_software_2015}،
\cite{p._miguel_review_2014}
و همچنین
\cite{sommerville_software_2016}
که به بررسی و مقایسه تطبیقی مدل‌های کیفی پرداخته‌اند، به این نکته پی می‌بریم که صحبت از کیفیت و پژوهش در مورد مدل‌های کیفی از همان ابتدا و به صورت همزمان با پژوهش‌های مربوط به توسعه نرم‌افزار و متدولوژی‌ها مورد توجه بوده است.
در شکل
\ref{fig:qmodels}
ملاحظه می‌شود که از سال ۲۰۰۱، کم‌کم مدل‌های عام‌منظوره‌ای همچون مدل‌های مک‌کال و درومی
\LTRfootnote{Dromey}
کم‌رنگ‌تر شدند و شاهد معرفی شدن مدل‌های خاص‌منظوره بودیم.
\begin{figure}[H]
	\centering\includegraphics[width=10cm]{Resources/qmodels.PNG}
	\caption[خط زمانی ارائه مدل‌های کیفیتی]
	{خط زمانی ارائه برخی از مدل‌های کیفیتی
		\cite{p._miguel_review_2014}؛
		با پیچیده شدن نیازمندی‌ها و گسترش آن‌ها، دسته‌ای از مدل‌ها تحت عنوان مدل‌های کیفیتی
		\lr{Tailored}
		در شکل مشخص شده‌اند که به طور خاص و برای یک نیازمندی خاصی ساخته شدند، مدل‌های
		\lr{Basic}
		اما به عنوان پایه مدل‌های کیفیتی عمل می‌کندد و تقریبا تمامی مدل‌ها از روی این مدل‌های اصلی اقتباس شده‌اند.
	}
	\label{fig:qmodels}
\end{figure}
مدل‌های عام‌منظوره که در شکل با نام Basic شناخته می‌شوند، ابعاد کلی کیفیت نرم‌افزار را هدف قرار داده‌اند و تقریبا می‌توانند در هر نرم‌افزاری مورد استفاده قرار بگیرند؛ مدل‌های بعدی که ارائه شدند، روی ابعاد خاصی از سازمان یا محصول نرم‌افزاری تمرکز داشته‌اند. این مدل‌ها در نتیجه افزایش پیچیدگی محصولات نرم‌افزاری و فرایندهای سازمانی، برای استفاده در کاربردهای خاص و برای سازمان‌های خاص توسعه داده شدند
\cite{p._miguel_review_2014}.\\
در بررسی مدل‌های کیفی، مرجع
\cite{wagner_software_2013}
دسته‌بندی‌ای را ارائه داده است که بر اساس آن، مدل‌های کیفی را می‌توان به سه دسته سلسله‌مراتبی، مبتنی بر متامدل و همچنین مدل‌های ضمنی تقسیم‌بندی کرد که توضیحات هرکدام در ادامه به صورت مختصر قید شده است.
\subsection{مدل‌های سلسله‌مراتبی}
روش‌های Boehm
\cite{boehm_quantitative_1976}
و مک‌کال
\cite{mccall_factors_1977}
در ارائه مدل کیفی، تشابه زیادی باهم دارند؛ هر دو در خرد کردن مفهوم کیفیت، از یک روش سلسله‌مراتبی استفاده کردند و مطابق شکل
\ref{fig:heir}
کیفیت را به خصیصه‌های مشخصی (که از آن‌ها با نام فاکتورهای کیفیت یاد می‌شود) تقسیم کرده‌اند. این‌گونه مدل‌ها در طول زمان دچار تغییراتی شدند و تفاوت نحوه تقسیم‌بندی آن‌ها، تفاوت مدل‌ها را پدید آورده است.
به تعبیر واگنر
\cite{wagner_software_2013}
رویکرد این نوع مدل‌های کیفی، خرد کردن کیفیت به معیارهای قابل اندازه‌گیری و در نهایت اندازه‌گیری و مقایسه آن‌هاست. همچنین واگنر در بررسی خود، از نقدهایی همچون «مبهم بودن برخی از این تقسیم‌بندی‌ها و شفاف نبودن آن‌ها به طور کامل» یاد می‌کند که از عوامل مهم ناکارآمدی برخی از آن‌هاست؛ همچنین وی در سال ۲۰۱۲ این نکته را متذکر شد که تنها کمتر از ۲۸٪ سازمان‌های فعال در حوزه نرم‌افزار، از مدل‌های استاندارد در تضمین کیفیت فعالیت‌ها و محصولاتشان استفاده می‌کنند و ۷۱٪ این سازمان‌ها، مدل‌های کیفی خود را از روی این مدل‌های کیفی، گلچین کرده و شخصی‌سازی می‌کنند
\cite{wagner_software_2012}.
\begin{figure}[H]
	\centering\includegraphics[width=11	cm]{Resources/heir.PNG}
	\caption{ساختار مدل کیفی مک‌کال به عنوان یک مدل کیفی سلسله‌مراتبی
		\cite{al-qutaish_quality_2010}
	}
	\label{fig:heir}
\end{figure}
همانطور که در شکل
\ref{fig:wagner}
مشاهده می‌شود، طی این بررسی، ۷۹ سازمان مدل‌های کیفیتی شخصی‌سازی‌شده توسط سازمان خود را در اولویت قرار داده و از آن‌ها استفاده می‌کنند. طبق اظهارات این بررسی و همچنین بسیاری از منابع دیگر همچون
\cite{pressman_software_2015} و
\cite{sommerville_software_2016}،
نیاز برای شخصی‌سازی مدل‌های کیفیتی استاندارد وجود دارد. چرا که این مدل‌های سلسله‌مراتبی، به صورت تجریدی بیان شده‌اند و نیازمند دقیق شدن روی متریک‌ها و روش‌های اندازه‌گیری هر متریک هستند.
\begin{figure}[H]
	\centering\includegraphics[width=11cm]{Resources/wagner.PNG}
	\caption[بررسی انواع مدل‌ها/رویه‌های کیفیتی استفاده شده در سازمان‌ها]
	{بررسی انواع مدل/رویه‌های کیفیتی استفاده شده در سازمان‌ها
		\cite{wagner_software_2012}؛
		ملاحظه می‌شود که ۸۹ شرکت (اکثریت) فعال در حوزه فناوری اطلاعات در آلمان، به منظور تضمین کیفیت محصولات خود، مدل کیفیتی مختص به خود را توسعه داده‌اند. اما در توسعه این مدل‌ها، همواره مدل‌های اصلی مورد اقتباس واقع شده‌اند و از آن‌ها استفاده شده است.
	}
	\label{fig:wagner}
\end{figure}
\subsection{مدل‌های مبتنی بر متامدل}
با آشکار شدن این نیاز که می‌بایست مدل‌های پایه‌ای را بیشتر شفاف‌سازی کرد و آن‌ها را بر نیازمندی‌ها تطبیق بیشتری داد، ایده ارائه متامدل‌ها مطرح شد. متامدل در اصل مدلی از یک مدل کیفی است؛ قواعد و ساختارهایی که برای توصیف دقیق یک مدل کیفی نیاز داریم (همچون متریک‌ها و نحوه اندازه‌گیری آن‌ها)، توسط متامدل تعیین می‌گردند
\cite{deissenboeck_software_2009}.
به عبارت دیگر، توصیف اینکه چگونه یک مدل کیفی می‌تواند بر نیازمندی‌ها منطبق شود، به عهده متامدل است
\cite{wagner_software_2013}.
درومی به عنوان مثال، در سال ۱۹۹۵، متامدل نسبتا مفصلی ارائه داد که ذیل آن، میان مولفه‌های محصول نرم‌افزاری (که باید حامل کیفیت باشند - مانند کد منبع نرم‌افزار) و ویژگی‌های عملیاتی نرم‌افزار تفاوت و تمایز قائل شد
\cite{dromey_model_1995}.
\subsection{مدل‌های آماری و ضمنی}
این مدل‌ها سعی در ترجمه مفهوم اطمینان‌پذیری سخت‌افزار و استفاده آن در حوزه نرم‌افزار را دارند. ایده اصلی استفاده از این مدل‌ها، مشاهده خرابی‌ها در طول زمان و پیش‌بینی روند رخ دادن خرابی‌ها در آینده است. به منظور دستیابی به خصیصه‌های کیفیتی مشخصی که برای نیازهای توسعه محور بیان شده،در برخی از مدل‌ها سعی شده از داده‌های آماری برای به دست آوردن برخی از ویژگی‌ها و متریک‌ها استفاده شود.\\
به عنوان مثالی برای این نوع مدل‌ها می‌توان به مدل‌های
\textit{رشد اعتمادپذیری}\LTRfootnote{Reliability Growth Models}
\cite{musa_software_2004}
اشاره کرد. مدل‌هایی که از الگوریتم‌ها و روش‌های یادگیری ماشین برای تخمین موارد مختلف، از قبیل مولفه‌های آسیب‌پذیر و یا غیر کارا و مولفه‌هایی که دارای برخی از ویژگی‌های کیفی خاص نیستند نیز از این نوع‌اند که نمونه‌ای از این مدل‌ها در مرجع
\cite{neuhaus_predicting_2007}
تحت عنوان Vulture یاد شده؛ این مدل، از روش‌های یادگیری ماشین و از یک پایگاه دانش آسیب‌پذیری استفاده می‌کند تا در طول زمان و با گسترش نرم‌افزار، بتواند مولفه‌های آسیب‌پذیر نرم‌افزار را پیش‌بینی کند.\\
همچنین به تعبیر مرجع‌های
\cite{sommerville_software_2016}
و
\cite{wagner_software_2013}،
ابزارها و روش‌های مرور، داشبوردهای مدیریتی و مصورسازی داده، ابزارهای شناخت الگوی رخ‌داد خطا در کد منبع نرم‌افزار و چک‌لیست‌ها، که شاید در ظاهر به طور مستقیم ارتباطی با مدل‌های کیفی نداشته باشند، اما در نهایت به یک یا چند متریک کیفی در ذیل یک مدل کیفی ختم می‌شوند؛ این اشاره به مدل‌های کیفی به صورت ضمنی و غیرصریح بوده و اغلب به طور دقیق ارتباط خود با مدل‌ها را مشخص نکرده‌اند. در نتیجه‌ی ‌تمام موارد ذکر شده، تضمین و کنترل کیفیت به واسطه این سازوکارهای غیرصریح و مدل‌هایی که به طور ضمنی مطرح هستند، منجر به پیچیدگی بیشتر و سختی کار خواهد شد.

\section{تمرکز بر استفاده‌پذیری روی مدل‌های کیفیتی}
در جدول
\ref{tab:models}\RTLfootnote{
	متریک‌های ذکر شده در جدول، همگی ترجمه شده عبارات لاتین هستند و در صورت داشتن ابهام در مورد هرکدام می‌توان به
	\hyperref[sec:glossary]{واژه‌نامه}
	رجوع کرد.
}
مقایسه‌ای تطبیقی میان مدل‌های مطرح از سال ۱۹۷۰ تا ۲۰۱۱ انجام شده است که در انجام این مقایسه، به طور خاص، روی خصیصه استفاده‌پذیری این مدل‌ها تمرکز داشتیم. مدل‌های عام‌منظوره در کنار سایر مدل‌ها به مقایسه درآمده‌اند تا خصیصه‌های استفاده‌پذیری در هرکدام از آن‌ها بررسی شود؛ مدل‌های خاص‌منظوره به خاطر نیاز سازمان خاصی به وجود آمده‌اند که مشتریان مخصوص به خود را داشتند که در صورت تعویض محصول و مشتری و استفاده مدل مفروض در یک سازمان دیگر، الزاما به جواب بهینه منتهی نخواهد شد. در حقیقت ویژگی اصلی مدل‌های کیفیتی مبتنی بر متامدل و دلیل گستردگی آن‌ها، متفاوت بودن نیازهای مشتریان و شرایط سازمان‌هاست
\cite{sommerville_software_2016}.\\
ارائه این مدل‌ها تا سال ۲۰۱۳ یکی از موضوعات پرطرفدار و داغ تحقیقاتی بوده اما به طور تدریجی و از سال ۲۰۱۳، با داشتن مراجعی همچون
\cite{albert_measuring_2013}
می‌توان مطالعه استفاده‌پذیری و رسیدن به یک محصول استفاده‌پذیر را با چک‌لیست‌ها و مطالعات آزمایشگاهی نیز تامین کرد. به خصوص که مرجع
\cite{wagner_software_2012}
از کسب‌وکارهای زیادی نام می‌برد که از هیچ‌کدام از مدل‌های کیفیتی ارائه شده پیشین استفاده نمی‌کنند و از قضا سودآوری زیادی نیز دارند و موفقیت محصولاتشان از سایر رقبا بیشتر است.
\begin{longtable}[c]{|c|c|c|C{6.8cm}|c|}
	\caption[مقایسه تطبیقی مدل‌های کیفیتی ارائه شده با تمرکز بر استفاده‌پذیری]
	{مقایسه تطبیقی مدل‌های کیفیتی ارائه شده با تمرکز بر استفاده‌پذیری}
	\label{tab:models} \\
	\hline
	\multicolumn{5}{| c |}{مقایسه تطبیقی مدل‌های کیفیتی}\\
	\hline
	ردیف & مدل کیفیتی & سال ارائه & متریک‌های استفاده‌پذیری & مرجع \\
	\hline
	\endfirsthead
	
	\hline
	\multicolumn{5}{| c |}{ادامه جدول \ref{tab:models}}\\
	\hline
	ردیف & مدل کیفیتی & سال ارائه & متریک‌های استفاده‌پذیری & مرجع \\
	\hline
	\endhead
	\multicolumn{5}{| c |}{ادامه جدول در صفحه بعد...}\\
	\hline
	\endfoot
	
	\hline
	\endlastfoot
	
	1 & McCall & 1970 & Operability, Training, Communicativeness & \cite{mccall_factors_1977} \\ \hline
	2 & Boehm & 1976 & Portability, Maintainability & \cite{boehm_quantitative_1976} \\ \hline
	3 & \lr{IEEE 1061} & 1990 & Comprehensibility, Ease of Learning, Communicativeness & \cite{radatz_ieee_1990} \\ \hline
	4 & Shackel & 1991 & Effectiveness, Learnability, Flexibility, Subjectively Pleasing & \cite{shackel_usability-context_1991} \\ \hline
	5 & Bevan & 1991 & Type of Product, Type of User, Ease of Use, Acceptability & \cite{bevan_what_1991} \\ \hline
	6 & FURPS & 1992 & Human Factors, Aesthetic, Documentation of the user, Material of Training & \cite{grady_practical_1992} \\ \hline
	7 & Nielsen & 1994 & Learnability, Efficiency, Memorability, Errors, Satisfaction & \cite{nielsen_usability_1994} \\ \hline
	8 & \lr{ISO 9126} & 2001 & Understandability, Learnability, Operability, Attractiveness, Usability Compliance & \cite{organization_iso/iec_1991}  \\ \hline
	10 & Bertoa & 2002 & Understandability, Learnability, Operability & \cite{bertoa_quality_2002} \\ \hline
	11 & Georgiadou & 2003 & Support, Learnability, Documentation Update, Online Help, Consistency & \cite{georgiadou_gequamogeneric_2003} \\ \hline
	13 & Abran & 2003 & Efficiency, Effectiveness, Satisfaction, Learnability, Security & \cite{abran_usability_2003} \\ \hline
	14 & Bass & 2003 & Modifiability, Scalability, Reusability, Performance, Security & \cite{bass_linking_2003} \\ \hline
	15 & Schneiderman & 2005 & Time to learn, Speed of Performance, Rate of Errors by users, Retention over time, Subjective Satisfaction & \cite{shneiderman_designing_2004}  \\ \hline
	16 & Rawashdeh & 2006 & Understandability, Learnability, Operability, Complexity & \cite{rawashdeh_new_2006} \\ \hline
	17 & \lr{ISO 25010} & 2008 & Approprateness, Recognizability, Learnability, Operability, User Error Protection, User Interface Aesthetics, Accessibility & \cite{noauthor_iso_nodate}  \\ \hline
	18 & Alvaro & 2010 & Understandability, Learnability, Operability & \cite{alvaro_quality_2005, alvaro_software_2010} \\ \hline
	19 & Alonso-Rios و بقیه & 2010 & Knowability, Operability, Efficiency, Robustness, Safety, Subjective Satisfaction & \cite{alonso-rios_usability:_2009} \\ \hline
	20 & Dubey و بقیه & 2011 & Effectiveness, Efficiency, Satisfaction, Learnability & \cite{kumardubey_usability_2012} \\ \hline
\end{longtable}
شایان ذکر است که در همه مدل‌های ذکر شده، الزاما به استفاده‌پذیری به عنوان یک خصیصه اصلی در محصول اشاره نشده است؛ در بعضی از مدل‌ها همچون
\lr{ISO 25010}
استفاده‌پذیری یکی از سطوح اصلی بوده، در اولین سطح سلسه‌مراتبی مدل قرار داشته و جزئی از محصول نهایی است و در برخی دیگر همچون
\lr{Boehm}،
به طور صریح و مشخص به استفاده‌پذیری اشاره‌ای نشده است اما در فرآیندهای توسعه محصول روی آن توجه زیادی وجود دارد.\\
همچنین از بررسی مدل‌های کیفیتی مختلف که صرفا برای توسعه سامانه‌های مبتنی بر وب این نتیجه برمی‌آید که هر متامدل می‌بایست در زمینه مربوط به خود مورد استفاده قرار گیرد و نه جای دیگر
\cite{noauthor_measuringu:_2018}.
نتیجه پیشین به این معنی است که در توسعه سامانه‌های مبتنی بر وب، محدوده کاربران، دانش قبلی آن‌ها، تخصص هرکدام، سن و سایر متغیرهای غیرقابل کنترل توسط توسعه‌دهنده نیز در استفاده‌پذیر بودن این سامانه مبتنی بر وب تاثیرگذار است؛ بنابراین در طراحی رابط کاربری هر سامانه مبتنی بر وب، می‌بایست به این نکات نیز توجه داشت و از آخرین توصیه‌های مربوط به توسعه این نوع سامانه‌ها استفاده کرد
\cite{albert_measuring_2013}.
از جمله این توصیه‌ها و پیشنهاد‌های طراحی، توصیه‌های گوگل برای ساخت سامانه‌های کاربردی مبتنی بر وب پیشرو\LTRfootnote{Progressive Web Applications}
\cite{noauthor_progressive_nodate}
است که در سال ۲۰۱۷ مطرح شده و طبق بررسی‌های انجام شده آینده‌ای روشن در انتظار این نوع از سامانه‌های کاربردی است.

\section{مطالعه استفاده‌پذیری و ارزیابی تجربه کاربری}
متریک‌هایی که در مطالعه استفاده‌پذیری و به طور خاص هنگام بررسی تجربه کاربری، اندازه‌گیری می‌شوند و مورد سنجش قرار می‌گیرند داده‌هایی را به دست ما می‌دهند که در بررسی و استفاده از این داده‌ها می‌توان دو رویکرد کلی داشت
\cite{albert_measuring_2013}:
ارزیابی خرد و ارزیابی کلان\RTLfootnote{
	در سال ۱۹۸۶ و طی مقاله‌ای با عنوان «نقش ارزیابی مستمر و چرخشی در طراحی سیستم‌ها برای کیفیت»
	\cite{hewett_role_1986}،
	دو اصطلاح برای مطالعه تجربه کاربری و استفاده‌پذیری
	\lr{Formative}
	و
	\lr{Summative}
	مورد استفاده قرار گرفتند که هر دو از مفاهیم کلاس درسی برداشت شده‌اند؛ یک ارزیابی مستمر
	(\lr{Formative})
	به معنی پرسیدن سوال در سر کلاس درس توسط معلم بوده و به صورت تدریجی و خرد خرد است؛ در حالی که یک ارزیابی  کلان
	(\lr{Summative})
	بررسی‌ای است که در انتهای هر بازه (مثلا هنگام امتحانات پایان‌ترم) و با برگزاری آزمونی خاص، ارزیابی‌ها انجام می‌شوند. در مقاله ذکر شده همچنین این مورد مطرح می‌شود که که ارزیابی مستمر و خرد برای رسیدن به دقت بالا در برآورد نیازهای مشتری، بهتر است؛ چرا که طبق مدل تشدید خرابی و خطا که پیش‌تر بررسی شد، خرابی‌ها هرچه کمتر بوده و نیازمندی‌های مشتری از همان ابتدا در نظر گرفته شوند و برآورده شوند، کیفیتی بیشتر با صرف هزینه‌ای کمتر خواهیم داشت.
}.\\
\begin{figure}
	\centering\includegraphics[width=10cm]{Resources/assessment.PNG}
	\caption{مثالی از دو نوع ارزیابی مختلف خرد و کلان
		\cite{noauthor_formative_nodate}
	}
	\label{fig:assessment}
\end{figure}
همانطور که در شکل
\ref{fig:assessment}
دیده می‌شود، می‌توان این دو نوع ارزیابی را به  چشیدن غذایی بدیل کرد که توسط آشپز و مشتری انجام می‌شوند؛ آشپز در فرآیند پختن غذا به طور مرتب ممکن است غذا را بچشد تا در نهایت خروجی مطلوبی به دست مشتری برسد و غذا از کیفیت لازم برخوردار باشد. در حالی که مشتری در نهایت، محصول نهای را مشاهده می‌کند و صرفا نظر خود در مورد آن غذا و یا کیفیت رستوران را اعلام می‌کند. ذکر این نکته  در همین‌جا خالی از لطف نیست که به وضح می‌توان دریافت که هزینه اعمال تغییرات در صورت درخواست مشتری از آشپز زیاد خواهد بود؛ به طور مشابهی، در صورت عرضه محصول نرم‌افزاری، هزینه تعمیر یک خرابی به مراتب بیشتر از مرور در حین تولید است.
\subsection{ارزیابی خرد}
در یک مطالعه استفاده‌پذیری با رویکرد ارزیابی خرد، محقق به طور مستمر و به صورت دوره‌ای، محصول نهایی را مورد بررسی قرار می‌دهد و در تمامی مراحل تولید نواقص آن را سنجیده و کشف می‌کند و پیشنهاداتی برای رفع آن نواقص ارائه می‌دهد؛ این روند  تا آن جا ادامه پیدا می‌کند که نهایتا یک محصول تقریبا ایده‌آل و یا یک محصول خوب به اندازه کافی\LTRfootnote{
	Good Enough
}
به دست آید. درواقع هدف در این نوع مطالعه هدف بهبود مستمر و رفع ایرادات محصول قبل از عرضه نهایی آن است؛ در نتیجه با بررسی فرآیندهای نرم‌افزاری و همچنین مطالعاتی از قبیل
\cite{sommerville_software_2016}،
\cite{krug_dont_2000} و
\cite{albert_measuring_2013}
به نظر می‌رسد که هرچه ارزیابی خرد زودتر رخ دهد، تاثیر بیشتری روی محصول نهایی و افزایش کیفیت آن خواهد داشت.\\
با اتخاذ این رویکرد، برخی از سوالاتی که می‌توان در فرایند طراحی پرسید عبارتند از:
\begin{itemize}
	\item 
	مهم‌ترین مواردی که کاربران را از رسیدن به اهدافشان منع می‌کند و یا به عدم کارایی آن‌ها می‌شود چیست؟
	\item 
	نقاط قوت و ضعف محصول از نقطه نظر کاربران چیست؟
	\item 
	اشتباهات متداول کاربران هنگام کار با محصول حول چه مواردی است؟
	\item 
	آیا بهبودهای مطرح شده توسط محققین تجربه کاربری، در هر نسخه از طراحی رابط کاربری، مورد استفاده و توجه قرار می‌گیرند؟
	\item 
	پس از عرضه نهایی محصول، چه مواردی در رابطه با استفاده‌پذیری به نظر می‌رسد که هنوز جای کار خواهد داشت؟
	
\end{itemize}
شایان ذکر است که در صورتی که فرصت اصلاح طراحی واسط کاربری وجود نداشته باشد، استفاده از این روش ارزیابی به نظر می‌رسد که کارایی چندانی نداشته باشد و بیشتر باعث هدررفت منابع شود.
\subsection{ارزیابی کلان}
در این روش، محقق همچون یک منتقد، محصول نهایی را از زوایای مختلف مورد بررسی قرار می‌دهد و حتی با محصول‌های دیگر مقایسه می‌کند تا نقدی بر آن وارد سازد. نکته حائز اهمیت این است که در اینجا محصول ارائه شده است و دیگر در فاز توسعه و تولید نیست. هدف از انجام این نوع ارزیابی، پی بردن به این نکته است که این محصول خاص چه‌قدر خوب می‌تواند به نیازمندی‌های کاربران پاسخ دهد و به چه میزان با آن‌ها هم‌جهت است. بر خلاف ارزیابی خرد، این روش، مبتنی بر اصول و قواعد و چک‌لیست‌های مشخصی است که در نهایت محصول با آن‌ها بررسی می‌شود. در مقایسه محصولات مختلف نیز مجددا این اصول و قواعد مبنا قرار می‌گیرند. با بررسی منابع مختلفی از قبیل
\cite{sommerville_software_2016} و
\cite{pressman_software_2015}
می‌توان به این نکته پی برد که سوالاتی از قبیل سوالات زیر بیشتر مناسب انجام این نوع ارزیابی هستند:
\begin{itemize}
	\item 
	آیا اهداف استفاده‌پذیری پروژه (مطرح شده در نیازمندی‌های پروژه) رعایت شده‌اند؟
	\item 
	استفاده‌پذیری کلی سیستم در چه سطحی است؟
	\item 
	نقاط ضعف و قوت محصول مورد نظر در مقایسه با سایر رقبا چیست و چگونه می‌توان در صورت داشتن ضعف، آن را ارتقا داد؟
	\item 
	آیا بهبودهای مطرح برای هر نسخه از نرم‌افزار، پس از عرضه نسخه جدید، اعمال می‌شوند؟
\end{itemize}
در نهایت فراموش نکنیم که همواره تغییر نیازمند صرف هزینه و زمان است؛‌ بنابراین در صورت استفاده از این روش ارزیابی می‌بایست در نظر داشت که برخی فعالیت‌های پسا ارزیابی نیز باید در پس ذهن مدیر پروژه باشد؛ چرا که ممکن است حتی در صورت نیاز پروژه‌ای برای برطرف کردن مشکلات استفاده‌پذیری یک سیستم تعریف شود که خود این پروژه هزینه‌بر باشد.
\subsection{اهداف کاربری}
سوالاتی همچون «آیا محصول مورد نظر نیاز روزانه کاربران را برآورده خواهد کرد و کاربران به طور متداول با این محصول نرم‌افزاری در ارتباط خواهند بود؟» و نیز «آیا کارایی کاربران و بهره‌وری آن‌ها در طول انجام یک وظیفه مشخص در هنگام کار با این نرم‌افزار مهم است؟ چگونه می‌توان آن را بهبود داد؟» به قسمتی از محصول توجه دارند که با نیازمندی‌های کاربر درگیر است. با بررسی‌ مراجعی همچون
\cite{albert_measuring_2013}،
\cite{noauthor_measuringu:_2018}،
\cite{hewett_role_1986} و
\cite{abran_usability_2003}
می‌توان به این نکته پی‌برد که همه این قبیل سوالات که به نیازهای ضمنی و نه الزاما صریح کاربر، در تعامل با رابط کاربری می‌پردازند، به دو متریک اساسی و قابل اندازه‌گیری از نیاز کاربران اشاره می‌کنند\RTLfootnote{
	البته باید توجه کرد که کارایی کاربر و رضایت کاربر در اینجا اشاره به دید کاربر به سامانه هدف دارند و نه اینکه متریک‌های اصلی مدل کیفیتی باشند. در اینجا کارایی و رضایت از دید کاربر و به عنوان نظر وی در مورد سامانه، مورد نظر هستند.
}: کارایی\LTRfootnote{Performance} و رضایت کاربر\LTRfootnote{Satisfaction}.
\paragraph{کارایی}
به عنوان یک متریک کیفیتی به طور خاص در رابطه با تجربه کاربری، به اندازه‌گیری توانایی کاربران در انجام وظایف مشخصی می‌پردازد؛ در این راستا، اندازه‌گیری‌های جنبی نیز اهمیت زیادی پیدا می‌کنند. از جمله این اندازه‌گیری‌ها می‌توان به موارد زیر اشاره کرد که به طور غیر مستقیم در کارایی تاثیرگذار هستند:
\begin{itemize}
	\item 
	زمان سپری شده برای انجام وظیفه
	\item 
	میزان تلاش برای انجام وظیفه (برای مثال تعداد کلیک‌ها و یا توان ذهنی مصرف شده)
	\item 
	زمانی که طول می‌کشد تا کاربر با وظیفه آشنا شود و بدون صرف تلاش خاصی آن را انجام دهد (یادگیری)
\end{itemize}
اندازه‌گیری‌های مربوط به متریک کارایی یک رابط کاربری، اهمیت زیادی دارند چرا که اگر کاربران نتوانند وظایف اصلی در رابطه با تعامل با سیستم را به درستی و با موفقیت به انجام برسانند، در عمل محصول نرم‌افزاری به شکست منتهی شده است و یا حداقل رابط کاربری خوبی ندارد و امکان تعامل موفق کاربر وجود نخواهد داشت.
\paragraph{رضایت}
درواقع نظر نهایی کاربر در مورد تعاملش با سیستم است؛ قضاوتی که کاربر در مورد سیستم و نحوه تعاملش با آن می‌کند می‌تواند با جملات مختلفی مانند «استفاده از آن سخت/آسان بود»، «گیج‌کننده/ساده بود» و ... بیان شود. البته که این تعبیرات غیردقیق هستند اما می‌توان با اعطای درجه‌های آزادی خاصی به کاربران، در حین تعامل با سیستم برخی از متریک‌ها را از آن‌ها به طور خوداعلامی\LTRfootnote{Self-reported Metrics} از کاربران گزارش عددی گرفت؛ چه بسا که به گفته مراجعی همچون
\cite{albert_measuring_2013}،
\cite{alonso-rios_usability:_2009} و
\cite{seffah_usability_2006}
این متریک‌ها در سامانه‌های کاربردی مبتنی بر وب - که هدف اصلی این پروژه هستند - بسیار مهم و تاثیرگذارند. اما باید به این نکته توجه کرد که در محدوده سامانه‌های مبتنی بر وب، رضایت کاربر الزاما همیشه همراه با کارایی حداکثری وی در تعامل با سامانه نیست؛ فاکتورهای بسیاری از قبیل زیبایی و وجود تکنولوژی‌های مختلف، بر این رضایت تاثیر مستقیم دارند و چه بسا که کاربری با رضایت حداکثری از یک سامانه استفاده کند ولی کارایی عملیات وی بسیار پایین باشد.\\
ویژگی‌های محصولی که کاربر با آن در تعامل است به عنوان یکی از اصلی‌ترین مدخل‌ها به بحث رضایت کاربری اهمیت دارند. در درجه دوم اما، کیفیت تعامل به عوامل دیگری همچون دانش قبلی کاربر و سن و جنسیت و غیره وابسته است. در ادامه مدلی برای تشخیص دادن ویژگی‌های محصول نهایی و اینکه کدام یک از آن‌ها و در چه شرایطی منجر به رضایت کاربری خواهند شد معرفی می‌شود.
\begin{figure}[H]
	\centering
	\includegraphics[width=7cm]{Resources/kano_model.PNG}
	\caption[تعبیری از مدل کانو]	{
		تعبیری از مدل کانو
		\cite{noauthor_kano_2018}
	}
	\label{fig:kano}
\end{figure}
\paragraph{مدل کانو}
که در سال ۱۹۸۴ و توسط آقای کانو
\cite{kano_attractive_1984}
برای تمییز دادن ویژگی‌های اشتیاق‌برانگیز و صریح و همچنین ویژگی‌های ضمنی و بایدی یک محصول، ارائه شد. طی این مدل، کیفیت محصول نهایی در گرو پنج دسته از نیازمندی‌های زیر است که رسیدن به هر دسته از این‌ها نیازمند اتخاذ سیاست‌های مختلف در طول ساخت محصول است:
\begin{itemize}
	\item 
	نیازمندی‌های بایدی: که مشتریان به طور ضمنی خواستار آن‌ها هستند و ممکن است صریحا بیان نشوند. به عنوان مثال اینکه یک سامانه کاربردی مبتنی بر وب همیشه با یک آدرس اینترنتی خاص
	\lr{URL}
	در دسترس باشد.
	\item
	نیازمندی‌های تک‌بعدی: که در صورت وجودشان کاربر احساس رضایتمندی و در صورت عدم وجودشان در محصول نهایی، کاربر احساس عدم رضایت از محصول را خواهد داشت. برای نمونه می‌توان به واکنش‌گرا بودن یک سامانه مبتنی بر وب روی پلتفرم موبایل اشاره کرد؛ که گفتنی است این روزها به یکی از ویژگی‌های اصلی موفقیت بسیاری از کسب‌وکارهای فعال در ایران تبدیل شده است.
	\item 
	نیازمندی‌های اشتیاق‌برانگیز: که در صورتی که به طور کامل پیاده‌سازی شوند، منجر به رضایتمندی کاربران خواهند شد ولی در صورت عدم پیاده‌سازی، رضایت کاربران از بین نخواهد رفت. به عنوان مثال اینکه یک سامانه پست الکترونیکی مبتنی بر وب\LTRfootnote{Webmail} در کنار لیست ایمیل‌های دریافتی، وضعیت آب‌وهوا و زمان فعلی و گزیده‌ای از اخبار را نشان دهد می‌تواند یک ویژگی اشتیاق برانگیز باشد.
	\item 
	نیازمندی‌های بی‌تفاوت: بودن و نبودنشان تفاوتی در رضایت مشتری نخواهد کرد. به عنوان مثال در بسیاری از پروژه‌های منتهی به یک سامانه کاربردی مبتنی بر وب، پلتفرم و زبان مورد استفاده برای توسعه سامانه، تفاوتی در رضایت مشتریان ایجاد نخواهد کرد.
	\item 
	نیازمندی‌های معکوس: به دسته‌ای از نیازمندی‌ها اشاره دارد که پرداختن بیش‌ازحد به آن‌ها باعث کاهش رضایت کاربران می‌شود. به عنوان مثال برخی از کاربران ممکن است از ابزارهایی که امکانات زیادی به آن‌ها در داشبورد مدیریتی می‌دهند خوششان بیاید و در مقابل برخی از کاربران از پیچیدگی بیش از حد ابزار گلایه کنند.
\end{itemize}
در شکل
\ref{fig:kano}
ملاحظه می‌شود که بسیاری از ویژگی‌های جذاب محصول که هنوز به عنوان نیازمندی مطرح هستند و هنوز پیاده‌سازی نشده‌اند و درنتجیه کاربر امکان انجام عملیات مورد نظر خود را ندارد، انگیزه‌ای برای ساخت سامانه هستند و پس از اینکه این ویژگی‌های عملیاتی (منحی قرمز رنگ) در محصول پدیدار می‌شوند، رضایت کاربران از محصول افزایش پیدا می‌کند؛ گرچه الزاما شاید این ویژگی‌ها، کارایی بالایی از دید کاربران نداشته باشند. به مرور زمان که فناوری پیشرفت می‌کند، نیازمندی‌های فعلی آهسته آهسته به باید‌های سامانه تبدیل می‌شوند (منحنی سبز رنگ).\\
همچنین در مرجع
\cite{sauerwein_kano_1996}
از خط آبی قابل مشاهده در شکل
\ref{fig:kano}،
به عنوان نیازمندی‌های تک‌بعدی یاد شده است که مشتری فقط به طور صریح و مشخص، این دسته از نیازمندی‌ها را مطرح می‌کند و بقیه نیازمندی‌ها معمولا به طور ضمنی مطرح می‌شوند؛ در نتیجه نقش تجربه در مهندسی نرم‌افزار و تولید سیستم‌های باکیفیت را می‌توان کمابیش مشاهده کرد؛ هرچه دانش بیشتری به بدیهیات و سهل‌های ممتنع موجود در نیازمندی‌ها داشته باشیم و اظهارمندی نیازمندی شفاف و مدون‌تری در دست باشد، محصول نهایی با کیفیت‌تر و هزینه و وقت صرف شده کم‌تر خواهد بود.\\
\begin{figure}[H]
	\centering
	\subfloat[تاثیر نیازمندی‌های اشتیاق‌بر‌انگیز]{
		\includegraphics[width=5.5cm]{Resources/kano_attractive.PNG}
	}
	\subfloat[تاثیر نیازمندی‌های تک‌بعدی]{
		\includegraphics[width=5.5cm]{Resources/kano_one.PNG}
	}
	\hspace{0mm}
	\subfloat[تاثیر نیازمندی‌های بایدی]{
		\includegraphics[width=5.5cm]{Resources/kano_must.PNG}
	}
	\subfloat[تاثیر نیازمندی‌های بی‌تفاوت]{
		\includegraphics[width=5.5cm]{Resources/kano_unimportant.PNG}
	}
	\hspace{0mm}
	\subfloat[تاثیر نیازمندی‌های معکوس]{
		\includegraphics[width=5.5cm]{Resources/kano_undesired.PNG}
	}
	\subfloat[تجمیع نیازمندی‌ها و تاثیرگذاری آن‌ها روی رضایت مشتریان]{
		\includegraphics[width=5.5cm]{Resources/kano_model2.PNG}
	}
	\caption[تفسیری از مدل کانو]
	{تفسیری از مدل کانو که نشان‌دهنده ارتباط رضایت کاربر و ویژگی‌های محصول است
		\cite{noauthor_leveraging_2012}؛
		نیازمندی‌های اشتیاق‌برانگیز کاربر، خواسته کاربران نیست ولی در صورت پیاده‌سازی موفق، موج بزرگی از رضایت را در بر خواهد داشت (آ). نیازمندی‌های تک‌بعدی در صورت عدم پیاده‌سازی، موجب نارضایتی خواهد بود (ب). نیازمندی‌های بایدی که درواقع خواسته اصلی کاربران بوده و باید به طور کامل پیاده‌سازی شوند تا رضایت حداقلی کسب شود (ج). نیازمندی‌های بی‌تفاوت نیز بود و نبودشان در محصول نهایی تفاوتی در رضایت مشتری ایجاد نخواهد کرد (د). نیازمندی‌های معکوس نشان داده‌اند که هرچه بیشتر به آن‌ها پرداخته شود، باعث بروز نارضایتی بیشتری خواهند شد (ه). در واقع هر ویژگی‌ای از محصول نهایی، مادامی که پیاده‌سازی نشده است، دچار انگیزش کاربر می‌شود. نیازمندی‌های تکبعدی پس از اینکه به کاملی پیاده‌سازی شد و کاربر از آن استفاده کرد، دچار افزایش رضایتمندی کاربر شده و در نهایت و پس از گذشت اندک زمانی، این ویژگی به یک ویژگی بایدی تبدیل می‌شود که کاربر حتی شاید به طور صریح به آن اشاره نکند ولی نبود آن در محصول باعث عدم رضایتمندی خواهد شد (و).
	}
	\label{fig:kano2}
\end{figure}
تعاریف انواع نیازمندی‌های مطرح شده توسط کانو به جهت اهمیتی که در شناخت خصیصه‌های کیفیتی این پروژه دارند، بار دیگر در شکل
\ref{fig:kano2}
ذکر شده است. مطابق این شکل، ممکن است با برهم زدن معاملات و همواره با تحویل دادن نیاز‌مندی‌های شگفت‌انگیز، مشتریان خود را غافل‌گیر کنیم و برای کسب رضایت موقتی، آن‌ها را از نیازمندی‌های اصلی دور کنیم، اما باید همواره در خاطر داشت که این استراتژی محکوم به شکست است چرا که محلول زمان در نهایت اشتیاق کاربران را در خود حل کرده و غافل‌گیری‌های دیروز تبدیل به باید‌های امروز خواهند بود و دیگر نمی‌توان ارزش افزوده‌ای نسبت به سایر رقیبان و یا نسبت به وضعیت دیروز سازمان خودمان، ارائه داد.\\
نتیجه نهایی از دو بحث پیشین در مورد رضایت کاربر از سامانه و کارایی کاربر در تعامل با سامانه، اینکه این دو خصیصه الزاما دارای همبستگی خاصی نیستند؛ ولی همواره باید در اندازه‌گیری استفاده‌پذیری مدنظر قرار بگیرند چرا که طبق تعریف استفاده‌پذیری، در یک سیستم استفاده‌پذیر، کاربر می‌بایست در نهایت از سیستم راضی بوده باشد و تجربه کاربری خوبی (کارایی در هنگام استفاده از محصول) داشته باشد.
\section{سناریوهای مطالعه استفاده‌پذیری و خصیصه‌های مطرح در هرکدام}
با مقایسه تطبیقی انجام شده در جدول
\ref{tab:models}
در رابطه با مدل‌های کیفیتی مختلف و همچنین با بررسی مراجعی همچون
\cite{wagner_software_2012}،
\cite{wagner_software_2013} و
\cite{albert_measuring_2013}
می‌توان نتیجه گرفت که در زمان انتخاب خصیصه‌های اندازه‌گیری استفاده‌پذیری در یک محصول (و نه الزاما یک محصول نرم‌افزاری)، می‌بایست به نکاتی از قبیل اهداف مطالعه استفاده‌پذیری\RTLfootnote{اینکه برای ارتقای محصول نهایی و ساخت یک نسخه دیگر است یا برای پیشگیری از بروز خرابی‌های بعدی و ...}، اهداف کاربری محصول\RTLfootnote{اینکه کاربر چه هدفی را از استفاده کردن از این محصول دنبال می‌کند}، فناوری‌ها و ابزارهای موجود برای جمع‌آوری داده و همچنین بودجه و زمان موجود برای تحقیق درباره استفاده‌پذیری، می‌بایست توجه کرد.\\
به تعبیر مرجع
\cite{albert_measuring_2013}،
در دنیای کنونی که نیازمندی‌ها بسیار گسترده و مفصل شده‌اند و جزئی‌ترین تغییرات در نیازمندی‌های مشتری ممکن است دامنه و حوزه مخاطبان یک سامانه را عوض کند و همچنین از آن‌جایی که در هر مطالعه استفاده‌پذیری، ویژگی‌ها و خصیصه‌های خاص آن مطالعه مدنظر قرار می‌گیرند -که الزاما با اهداف سایر مطالعات استفاده‌پذیری یکسان نیستند- نمی‌توان مجموعه‌ای از متریک‌های مشخص و شسته‌رفته‌ای برای اندازه‌گیری و سنجش استفاده‌پذیری در تمام سامانه‌های کاربردی مبتنی بر وب ارائه داد؛‌
در حقیقت، طبق ادعای مرجع
\cite{wagner_software_2012}
بیش از ۷۰٪ سازمان‌های فعال در حوزه فناوری اطلاعات، به منظور تامین و تضمین کیفیت در محصولات نرم‌افزاری خود، مدل‌های گلچین شده و سفارشی‌سازی شده خود را استفاده می‌کنند. بنابراین این پندار که برای تمامی سامانه‌های مبتنی بر وب می‌توان یک مدل کیفیتی ثابت ارائه کرد، غیرمنطقی به نظر می‌رسد. در عوض می‌توان با بررسی سرگذشت تاریخی مدل‌های کیفی و از طریق مقایسه تطبیقی آن‌ها در جدول
\ref{tab:models}
و نیز مطالعه مراجع مختلفی همچون
\cite{alonso-rios_usability:_2009}،
\cite{bass_linking_2003}،
\cite{bevan_what_1991}،
\cite{pressman_software_2015}،
\cite{sommerville_software_2016}
و
\cite{albert_measuring_2013}
که در زمینه تجربه کاربری و استفاده‌پذیری نتایج تحقیقات ارزشمندی را ارائه کرده‌اند، به این نتیجه دست یافت که دسته‌بندی‌های کلی‌ای از انواع مطالعات استفاده‌پذیری می‌توان ارائه نمود، که در هر دسته، با توجه به اهداف و نتایج مورد نیاز مطالعه، متریک‌های خاصی اهمیت بیشتری پیدا می‌کنند و می‌توان به آن‌ها پرداخت.\\
یکی از انواع متامدل‌ها که بر مبنای مدل‌های سلسه‌مراتبی و به طور خاص برای استفاده‌پذیری سامانه‌های کاربردی مبتنی بر وب ساخته شده است در مرجع
\cite{albert_measuring_2013}
ذکر شده است\RTLfootnote{
	البته تمامی خصیصه‌های مطرح برای استفاده‌پذیری در این مدل به طور کامل شرح و بسط داده نشده‌اند؛ چرا که به گفته ارائه دهنده، بسیاری از اقدامات به زمینه مورد مطالعه و تست وابسته بوده و در نتیجه مشخص کردن جزییات نهایی با تیم تضمین کیفیت (تست و ارزیابی استفاده‌پذیری) است. همچنین شایان ذکر است که هیچگاه منبع ذکر شده از خصیصه‌هایی که عنوان کرده به عنوان یک چارچوب و یا متامدل یاد نکرده است؛ اما به جهت اهمیت استدلال و بررسی‌های این منبع و همچنین ارجاعات زیاد به آن، چه در صنعت و ابزارهایی همچون
	\lr{UsabilityHub}
	و چه در دنیای تحقیقات، اهمیت این خصیصه‌های کیفیتی را به عنوان یک متامدل کیفیتی دو چندان می‌کند.
}.
با بررسی‌های انجام گرفته روی انواع مدل‌های کیفیتی عام و خاص منظوره که در موقعیت‌های مختلف ارائه شده‌اند، ده سناریوی مهم برای 
در ادامه ده سناریوی مطرح در مطالعه استفاده‌پذیری مطرح می‌شود که می‌توان عناوین هرکدام و متریک‌های مناسب هر سناریو را در جدول
\ref{tab:10usability}
مشاهده کرد. در ادامه به بررسی هرکدام از این ده سناریو می‌پردازیم.
\begin{table}[H]
	\caption[سناریوهای متداول مطالعه استفاده‌پذیری و متریک‌هایی که می‌توان برای هرکدام در نظر گرفت]{
		سناریوهای متداول مطالعه استفاده‌پذیری و متریک‌هایی که می‌توان برای هرکدام در نظر گرفت
		\cite{albert_measuring_2013}
	}
	\label{tab:10usability}
	\centering
	%	\resizebox{\textwidth}{!}{%
	\begin{tabular}{|C{5cm}|c|c|c|c|c|c|c|c|c|c|c|}
		\hline
		‏هدف و سناریوی مطالعه استفاده‌پذیری
		& 
		\rotatebox{90}{‏موفقیت‌آمیز بودن وظایف}
		& 
		\rotatebox{90}{زمان انجام وظایف}
		& 
		\rotatebox{90}{خطاها}
		& 
		\rotatebox{90}{کارایی}
		& 
		\rotatebox{90}{یادگیری‌پذیری}
		& 
		\rotatebox{90}{متریک‌های مبتنی بر موضوع}
		& 
		\rotatebox{90}{متریک‌های خوداعلامی}
		& 
		\rotatebox{90}{متریک‌های رفتاری و روانشناسانه}
		& 
		\rotatebox{90}{متریک‌های ترکیبی و مقایسه‌ای}
		& 
		\rotatebox{90}{متریک‌های وبسایت زنده}
		& 
		\rotatebox{90}{داده‌های مرتب}
		\\ \hline
		انجام یک تراکنش & × &  &  & × &  & × & × &  &  & × &  \\ \hline
		مقایسه محصولات & × &  &  & × &  &  & × &  & × &  &  \\ \hline
		ارزیابی استفاده مکرر از محصول & × & × &  & × & × &  & × &  &  &  &  \\ \hline
		ارزیابی پیمایش و معماری اطلاعات سامانه & × &  & × & × &  &  &  &  &  &  & × \\ \hline
		افزایش آگاهی &  &  &  &  &  &  & × & × &  & × &  \\ \hline
		کشف مشکل &  &  &  &  &  & × & × &  &  &  &  \\ \hline
		حداکثرسازی استفاده‌پذیری یک محصول حیاتی & × &  & × & × &  &  &  &  &  &  &  \\ \hline
		ایجاد تجربه کاربری مثبت &  &  &  &  &  &  & × & × &  &  &  \\ \hline
		ارزیابی تاثیرات تغییرات جزئی و نامحسوس &  &  &  &  &  &  &  &  &  & × &  \\ \hline
		مقایسه طراحی‌های مختلف & × & × &  &  &  & × & × &  & × &  &  \\ \hline
	\end{tabular}%
	%	}
\end{table}
نکته قابل تامل و مهم این است که در هر کدام از این ده سناریوی مطرح شده، الزاما به تمام ابعاد کیفیتی نگاه نمی‌شود؛ هرچیزی که باعث می‌شود استفاده‌پذیری تحت‌تاثیر قرار بگیرد اهمیت داشته و به جز آن بررسی نمی‌شود. حتی ممکن است در یک سناریو، مواردی از قبیل طراحی پیمایش و طراحی مولفه (رجوع شود به هرم طراحی سامانه‌های کاربردی مبتنی بر وب، شکل \ref{fig:pyramid}) نیز مورد بحث و بررسی قرار گیرند.
\subsection{انجام یک تراکنش}
برخی از مطالعات استفاده‌پذیری به جهت افزایش بهره‌وری بیشتر، بهبود و هموارتر شدن روند انجام یک تراکنش هستند؛ این تراکنش می‌تواند «تغییر گذرواژه»، «انجام یک خرید» و غیره باشد. اساسا یک تراکنش دارای یک نقطه آغاز و یک نقطه پایان مشخص و واضح است؛ مثلا در یک سایت خرید و فروش برخط کالا، قرار دادن یک قلم کالا در سبد خرید و اتمام خرید و تایید شدن سفارش، به ترتیب، نقاط شروع و پایان تراکنش هستند.
\subsection{مقایسه محصولات}
هدف از برخی مطالعات استفاده‌پذیری مقایسه بین یک نسخه از یک محصول و یک نسخه از محصول دیگر و یا نسخه‌های قبلی همان محصول، به منظور یافتن نقاط ضعف و قوت هر کدام در طراحی و استفاده‌پذیری است. بنابراین می‌توان گفت که با مقیاسه محصولات می‌توان ارزیابی خوبی از استفاده‌پذیری سامانه هدف، در مقایسه با رقبا داشت و پتانسیل‌های موجود برای افزایش استفاده‌پذیری را شناخت.
\subsection{ارزیابی استفاده مکرر از محصول}
بسیاری از محصولات و سامانه‌های نرم‌افزاری از جمله وب‌سایت‌های شبکه‌های اجتماعی، سامانه‌های اتوماسیون و ... برای استفاده مکرر در طول روز ساخته شده‌اند و برای رسیدن به همین هدف می‌بایست هم استفاده از آن‌ها آسان باشد و هم بهره‌وری زیادی داشته باشند؛ بنابراین در این سناریوی بررسی و با شمردن اهداف پیشین، به نظر می‌رسد که زمان صرف شده برای انجام یک وظیفه از جمله مهم‌ترین خصیصه‌ها برای اندازه‌گیری استفاده‌پذیری به شمار می‌آید.
\subsection{ارزیابی پیمایش و معماری اطلاعات سامانه}
به تعبیر مرجع
\cite{albert_measuring_2013}
این سناریو معروف‌ترین سناریو برای سنجش استفاده‌پذیری سامانه‌های مبتنی بر وب است. در انجام مطالعه استفاده‌پذیری طبق این سناریو، می‌توان مواردی همچون «اطمینان از اینکه کاربران حتما چیزی را که می‌خواهند پیدا می‌کنند»، «در صفحات مختلف به آسانی پیمایش می‌کنند» و مواردی از این دست را نیز جزوی از اهداف مطالعه دانست. به طور معمول این مطالعات شامل طرح‌های مفهومی قبل از پیاده‌سازی هستند که هدف اصلی در این پروژه نیز بررسی و سنجش استفاده‌پذیری این طرح‌های مفهومی است؛ درواقع در این طرح‌های مفهومی نحوه بدست آوردن اطلاعات و طراحی پیمایش و تجربه ابتدایی کاربری به قدری مهم است که باید قبل از هر چیزی، و در اولین وهله، تعیین تکلیف شوند.
از جمله خصیصه‌های مهم در این سناریو، موفقیت انجام وظیفه‌های مختلف است.
\subsection{افزایش آگاهی}
گاهی اوقات تغییرات در رابط‌های کاربری، فقط به جهت افزایش بهره‌وری کاربر نیست؛ بلکه هدف از اعمال تغییرات در سامانه، افزایش آگاهی نسبت به یک قسمت خاص است. این مسئله در مورد تبلیغات برخط درست است؛ علاوه بر آن، در مواردی هم که پتانسیل‌های کارکردی یک سامانه، به تمامی مورد استفاده قرار نمی‌گیرند نیز این مسئله صادق است.
\subsection{کشف مشکل}
در این نوع مطالعه هدف پیدا کردن مشکلات عمده در استفاده‌پذیری محصول است. گاهی اوقات به دلایل مختلفی از جمله سهل ممتنع بودن و یا مخفی بودن مشکلات استفاده‌پذیری در چشم توسعه‌دهنده، عملا تیم توسعه و تولید محصول قادر به کشف مشکلات استفاده‌پذیری محصول نیستند، در حالی که مشتریان از این نوع مشکلات گلایه می‌کنند. معمولا این نوع مطالعه‌ها انتهای باز دارند و نتیجه‌گیری قاطعی نمی‌شود از آن‌ها کرد.
\subsection{حداکثرسازی استفاده‌پذیری یک محصول حیاتی}
اگرچه تولیدکنندگان بررخی از محصولات و سامانه‌ها، همچون شبکه‌های اجتماعی، وبلاگ‌ها و ... به دنبال آسان کردن هرچه بیشتر نحوه تعامل کاربران و راحتی کار آن‌ها هستند، برخی از محصولات و سامانه‌های حساس
\texttt{باید}
استفاده‌پذیری بالا و راحتی استفاده داشته باشند. از جمله این سامانه‌ها می‌توان به سامانه‌های رای‌گیری، سامانه‌های خروج اضطراری هواپیماها و ... اشاره کرد. فلسفه وجودی سامانه‌های حساس، همچون سایر سامانه‌ها، این است که کاربر می‌بایست در مجموع چند کار بسیار محدود را با آن‌ها انجام دهد. اما تفاوت آن‌ها با سامانه‌های عادی در این است که کاربر می‌بایست در تعامل با سامانه‌های حساس و حیاتی، حتما وظایفش را با صددرصد موفقیت و اطمینان انجام دهد و در صورت عدم موفقیت در انجام و یا صرف زمان بیش‌از‌حد و یا وجود هر مشکل استفاده‌پذیری دیگری، ممکن است خسارت جانی و مالی وی را تهدید کند. بنابراین استفاده‌پذیری بالا در این سامانه‌ها می‌بایست طی آزمایش‌های مختلف و در انواع شرایط مختلف، و نه فقط به صورت محدود در فضای آزمایشگاه، سنجیده شده و اثبات شود.
\subsection{ایجاد تجربه کاربری مثبت}
در برخی از سامانه‌ها، اینکه تجربه کاربری و تجربه تعامل مثبتی با سامانه داشته باشد هدف اصلی سازمان تولیدکننده محصول است. در این سامانه‌ها معمولا ویژگی‌هایی از قبیل اشتیاق‌انگیزی در کاربر، اجین کردن احساسات و عواطف کاربر، مشغول کردن وی و همچنین کمی ایجاد اعتیاد به استفاده از سامانه در وی، از ویژگی‌های بارز و قابل مشاهده در محصول نهایی است.\\
برخی مطالعات استفاده‌پذیری به منظور افزایش این تجربه کاربری و تقویت این ویژگی‌ها در محصول نهایی انجام می‌شوند. به عنوان مثال یکی از مدیران اسبق فیسبوک، از طراحی معتادکننده این شبکه اجتماعی خبر می‌دهد
\cite{noauthor_sean_nodate}
که به گفته وی آگاهانه بوده و در جهت افزایش سوددهی این شرکت بوده است.
\subsection{ارزیابی تاثیرات تغییرات جزئی و نامحسوس}
گاهی تغییرات اعمال شده در طراحی رابط  کاربری و در کل در سامانه کاربردی به قدری کوچک‌اند که کاربر آن‌ها را حس نمی‌کند و نمی‌توان تاثیر این تغییرات را در رفتار کاربران سنجید. اما به تجربه
\cite{albert_measuring_2013}
می‌توان گفت که حتی تغییر اندازه قلم نوشته‌های رابط کاربری یک سامانه کاربردی مبتنی بر وب در یک شبکه اجتماعی با تعداد کاربران نسبتا زیاد، منجر به تغییرات  بزرگ و گاها هزینه‌بر در تجربه کاربری کاربران می‌شود. با مقدمه فوق، برخی از مطالعات استفاده‌پذیری پس از اعمال تغییراتی در محصول، با هدف سنجش تاثیر این تغییرات روی تجربه کاربری انجام می‌شوند.
\subsection{مقایسه طراحی‌های مختلف}
هدف در این نوع مطالعه، مقایسه طرح فعلی با سایر طراحی‌های موجود، و نه الزاما یک طرح، است. این مطالعه نیز از جمله مطالعه‌های مشهور استفاده‌پذیری است. به طور معمول و طبق بررسی‌های منبع
\cite{albert_measuring_2013}
این نوع مطالعه نیز در مراحل اولیه تولید و با در دست داشتن طرح‌های مفهومی و به منظور سنجش «خوب بودن» طرح فعلی در مقایسه با سایر طراحی‌ها انجام می‌شود. طی این مطالعه معمولا تیم‌های مختلف طراحی، نمونه‌های اولیه\LTRfootnote{Prototype}
خود را به کارزار مقایسه و ارزیابی وارد می‌کنند و با استفاده از خصیصه‌های از پیش تعیین شده‌ای شروع به سنجش و مقایسه آن‌ها می‌کنند.\\
چالش انجام این نوع مطالعات معمولا در مشابه بودن طراحی‌ها است؛ البته همین شباهت زیاد باعث می‌شود که بتوان نکات مثبت طراحی‌ها را الهام گرفت و در طراحی، آن‌ها را پیاده‌سازی کرد. درخواست انجام یک عمل مشخص در تمامی طراحی‌ها و نتایج برآمده از آن‌ها معمولا نمی‌تواند معیار مناسبی در این مطالعه باشد.
\section{روش‌های سنجش استفاده‌پذیری}
به لطف فناوری وب و شبکه، امروزه محدود به یک روش سنجش و ارزیابی نیستیم. همانطور که از بررسی منابعی همچون
\cite{agarwal_assessing_2002}،
\cite{albert_measuring_2013}،
\cite{krug_dont_2018} و
\cite{noauthor_measuringu:_2018}
برمی‌آید، می‌توان خصیصه‌ها و اندازه‌گیری‌های مربوط به آن‌ها را تقریبا از هر روشی که ارزیابی بعدی از این داده‌ها را تضمین کند، به دست آورد. جمله پیشین به این معنی است که در مطالعه، سنجش و ارزیابی استفاده‌پذیری، نه تنها محدود به مشاهدات و آزمایشات آزمایشگاهی\RTLfootnote{
	به منظور مطالعه بیشتر می‌توان به مرجع
	\cite{li_crowdsourced_2016}
	مراجعه نمود که مروری روی انگیزه‌ها، روش‌ها و چالش‌های جمع‌سپاری می‌کند و از جمع‌سپاری به عنوان یک راه جمع‌آوری داده برای انجام ارزیابی‌ها و سنجش‌های مختلف نام می برد؛ در این منبع همچنین به دفعات متعدد اثبات شده است که هزینه استفاده از جمع‌سپاری برای جمع‌آوری داده به مراتب از روش‌هایی همچون روش‌های آزمایشگاهی کمتر بوده و با استفاده از این روش می‌توان با صرف زمان و هزینه کمتر، به نتایج گسترده‌تری رسید.
}
(که به معنای تعداد کم شرکت‌کنندگان و صرف زمان و هزینه زیاد است) نیستیم، بلکه می‌توانیم از روش‌هایی همچون مطالعات برخط استفاده کرده و حجم زیادی از تحلیل را در زمان کمی به دست آوریم\RTLfootnote{
نکته قابل تامل از این نتیجه‌گیری این است که سازمان‌ها، کسب‌وکارها و شبکه‌های اجتماعی می‌توانند از قدرت کاربران خود استفاده کنند تا برای جمع‌آوری داده راهی هموارتر، که به دنبال آن سودآوری بیشتر خواهد آمد، داشته باشند. این به این معناست که همین پروژه می‌تواند در صورت پیاده‌سازی تجاری توسط یک سازمان دارای کسب‌وکاری از نوع
\lr{B2C (Business to Client)}
یک منبع درآمد اضافی باشد؛ کسب‌وکارهای نوپا
(\lr{Startups})
و سریس‌های مبتنی بر وب جدید را در نظر بگیرید. همه این کسب‌وکارها که سرویس خود را با کمک سامانه‌های مبتنی بر وب ارائه می‌دهند، همگی مایل‌اند که با صرف حداقل هزینه، به بهترین محصول برای شروع کسب‌وکار خود برسند. سازمان ارائه‌دهنده سرویس سنجش استفاده‌پذیری برای کسب‌وکارهای نوپا می‌تواند با استفاده از کاربران خود و با بهره‌گیری از روش‌های جمع‌سپاری و خرد‌کردن تست‌ها به میکرووظایف، داده‌های مربوط برای تست و سنجش استفاده‌پذیری محصولات نرم‌افزاری کسب‌وکارهای نوپا را فراهم کند. بنابراین این پروژه می‌تواند به عنوان یک طرح کسب‌وکاری
(\lr{Business Plan})
اولیه برای سازمان‌های درگیر با کاربران نهایی
(\lr{B2C})
و همچنین کسب‌وکارها
(\lr{B2B - Business to Business})
باشد.
}.
\section{ابزارهای سنجش استفاده‌پذیری}
از جمله 





